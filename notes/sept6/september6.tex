% Created 2023-09-06 Wed 10:25
% Intended LaTeX compiler: pdflatex
\documentclass[presentation]{beamer}
   \usepackage[T1]{fontenc}
   \setbeamerfont{frametitle}{size=\large}
   \setbeamertemplate{footline}{
   \begin{flushright}
   \insertframenumber/\inserttotalframenumber\hspace{2mm}\mbox{}
   \end{flushright}}
   \usepackage{libertine}
   \usepackage[varqu]{zi4}
   \usepackage[libertine]{newtxmath}
   %%\usepackage[small]{eulervm}
   %%\usepackage[{libertinust1math}
   \usepackage[T1]{fontenc}
   \renewcommand*\familydefault{\sfdefault}
   \usepackage{tipa}
   \usepackage{adjustbox}
   \usepackage{multirow}
   \usepackage{multicol}
   \usepackage[backend=bibtex,uniquename=false,dashed=false,date=year,url=false,isbn=false,style=authoryear-comp]{biblatex}
\addbibresource{/home/juliangrove/Documents/projects/FACTS.lab/ur-comp-sem-2023/ur-comp-sem-2023.bib}
   \usepackage{expex}
   \usepackage{stmaryrd}
   \usepackage{stackrel}
   \usepackage{stackengine}
   \usepackage{relsize}
   \usepackage{amsmath}
   \usepackage{mathtools}
   \usepackage{dsfont}
   \usepackage{fixltx2e}
   \usepackage{graphicx}
   \usepackage{xcolor}
   \usepackage[normalem]{ulem}
\usetheme{metropolis}
\author{Julian Grove}
\date{September 6, 2023}
\title{Haskell: variables, data types, patterns, and recursion}
\usepackage{tikz}
\usetikzlibrary{arrows,arrows.meta,automata}
\usepackage[linguistics]{forest}
\forestset{ downroof/.style={ for children={ if n=1{ edge path'={ (.parent first) -- (!u.parent anchor) -- (!ul.parent last) -- cycle }}{no edge} } } }
\definecolor{highlight}{RGB}{7,102,120}
\definecolor{dim}{RGB}{235,219,178}
\definecolor{dimmer}{RGB}{150,130,50}
\newcommand{\ct}[1]{\textsf{#1}}
\newcommand{\eval}[1]{#1^\lightning}
\newcommand\observe{\mathit{observe}}
\newcommand\reset{\mathit{reset}}
\newcommand\factor{\mathit{factor}}
\def\divd{\ |\ }
\newcommand{\maptyp}[1]{\overline{#1}}
\usepackage[verbose]{newunicodechar}
\newunicodechar{¬}{\ensuremath{\neg}} %
\newunicodechar{¹}{^1}
\newunicodechar{²}{^2}
\newunicodechar{³}{^3}
\newunicodechar{·}{\ensuremath{\cdot}}
\newunicodechar{×}{\ensuremath{\times}} %
\newunicodechar{÷}{\ensuremath{\div}} %
\newunicodechar{ᵅ}{^α}
\newunicodechar{ᵝ}{^β}
\newunicodechar{ˠ}{^γ}
\newunicodechar{ᵃ}{^a}
\newunicodechar{ᵇ}{^b}
\newunicodechar{ᶜ}{^c}
\newunicodechar{ᵈ}{^d}
\newunicodechar{ᵉ}{^e}
\newunicodechar{ᶠ}{^f}
\newunicodechar{ᵍ}{^g}
\newunicodechar{ʰ}{^h}
\newunicodechar{ⁱ}{^a}
\newunicodechar{ʲ}{^j}
\newunicodechar{ᵏ}{^K}
\newunicodechar{ˡ}{^l}
\newunicodechar{ᵐ}{^m}
\newunicodechar{ⁿ}{^n}
\newunicodechar{ᵒ}{^o}
\newunicodechar{ᵖ}{^p}
\newunicodechar{ʳ}{^r}
\newunicodechar{₁}{_1}
\newunicodechar{₂}{_2}
\newunicodechar{₃}{_3}
\newunicodechar{₄}{_4}
\newunicodechar{ₐ}{_a}
\newunicodechar{ₑ}{_e}
\newunicodechar{ₕ}{_h}
\newunicodechar{ᵢ}{_i}
\newunicodechar{ₙ}{_n}
\newunicodechar{ᵣ}{_r}
\newunicodechar{ₛ}{_s}
\newunicodechar{̷}{\not} %
\newunicodechar{Γ}{\ensuremath{\Gamma}}   %
\newunicodechar{Δ}{\ensuremath{\Delta}} %
\newunicodechar{Η}{\ensuremath{\textrm{H}}} %
\newunicodechar{Θ}{\ensuremath{\Theta}} %
\newunicodechar{Λ}{\ensuremath{\Lambda}} %
\newunicodechar{Ξ}{\ensuremath{\Xi}} %
\newunicodechar{Π}{\ensuremath{\Pi}}   %
\newunicodechar{Σ}{\ensuremath{\Sigma}} %
\newunicodechar{Φ}{\ensuremath{\Phi}} %
\newunicodechar{Ψ}{\ensuremath{\Psi}} %
\newunicodechar{Ω}{\ensuremath{\Omega}} %
\newunicodechar{α}{\ensuremath{\mathnormal{\alpha}}}
\newunicodechar{β}{\ensuremath{\beta}} %
\newunicodechar{γ}{\ensuremath{\mathnormal{\gamma}}} %
\newunicodechar{δ}{\ensuremath{\mathnormal{\delta}}} %
\newunicodechar{ε}{\ensuremath{\mathnormal{\varepsilon}}} %
\newunicodechar{ζ}{\ensuremath{\mathnormal{\zeta}}} %
\newunicodechar{η}{\ensuremath{\mathnormal{\eta}}} %
\newunicodechar{θ}{\ensuremath{\mathnormal{\theta}}} %
\newunicodechar{ι}{\ensuremath{\mathnormal{\iota}}} %
\newunicodechar{κ}{\ensuremath{\mathnormal{\kappa}}} %
\newunicodechar{λ}{\ensuremath{\mathnormal{\lambda}}} %
\newunicodechar{μ}{\ensuremath{\mathnormal{\mu}}} %
\newunicodechar{ν}{\ensuremath{\mathnormal{\mu}}} %
\newunicodechar{ξ}{\ensuremath{\mathnormal{\xi}}} %
\newunicodechar{π}{\ensuremath{\mathnormal{\pi}}}
\newunicodechar{π}{\ensuremath{\mathnormal{\pi}}} %
\newunicodechar{ρ}{\ensuremath{\mathnormal{\rho}}} %
\newunicodechar{σ}{\ensuremath{\mathnormal{\sigma}}} %
\newunicodechar{τ}{\ensuremath{\mathnormal{\tau}}} %
\newunicodechar{φ}{\ensuremath{\mathnormal{\phi}}} %
\newunicodechar{χ}{\ensuremath{\mathnormal{\chi}}} %
\newunicodechar{ψ}{\ensuremath{\mathnormal{\psi}}} %
\newunicodechar{ω}{\ensuremath{\mathnormal{\omega}}} %
\newunicodechar{ϕ}{\ensuremath{\mathnormal{\phi}}} %
\newunicodechar{ϕ}{\ensuremath{\mathnormal{\varphi}}} %
\newunicodechar{ϵ}{\ensuremath{\mathnormal{\epsilon}}} %
\newunicodechar{ᵏ}{^k}
\newunicodechar{ᵢ}{\ensuremath{_i}} %
\newunicodechar{ }{\quad}
\newunicodechar{†}{\dagger}
\newunicodechar{ }{\,}
\newunicodechar{′}{\ensuremath{^\prime}}  %
\newunicodechar{″}{\ensuremath{^\second}} %
\newunicodechar{‴}{\ensuremath{^\third}}  %
\newunicodechar{ⁱ}{^i}
\newunicodechar{⁵}{\ensuremath{^5}}
\newunicodechar{⁺}{\ensuremath{^+}} %% #+beamer_header: \newunicodechar{⁺}{^+}
\newunicodechar{⁻}{\ensuremath{^-}} %%
\newunicodechar{ⁿ}{^n}
\newunicodechar{₀}{\ensuremath{_0}} %
\newunicodechar{₁}{\ensuremath{_1}} % #+beamer_header: \newunicodechar{₁}{_1}
\newunicodechar{₂}{\ensuremath{_2}} % #+beamer_header: \newunicodechar{₂}{_2}
\newunicodechar{₃}{\ensuremath{_3}}
\newunicodechar{₊}{\ensuremath{_+}} %%
\newunicodechar{₋}{\ensuremath{_-}} %%
\newunicodechar{ₙ}{_n} %
\newunicodechar{ℂ}{\ensuremath{\mathbb{C}}} %
\newunicodechar{ℒ}{\ensuremath{\mathscr{L}}}
\newunicodechar{ℕ}{\mathbb{N}} %
\newunicodechar{ℚ}{\ensuremath{\mathbb{Q}}}
\newunicodechar{ℝ}{\ensuremath{\mathbb{R}}} %
\newunicodechar{ℤ}{\ensuremath{\mathbb{Z}}} %
\newunicodechar{ℳ}{\mathscr{M}}
\newunicodechar{⅋}{\ensuremath{\parr}} %
\newunicodechar{←}{\ensuremath{\leftarrow}} %
\newunicodechar{↑}{\ensuremath{\uparrow}} %
\newunicodechar{→}{\ensuremath{\rightarrow}} %
\newunicodechar{↔}{\ensuremath{\leftrightarrow}} %
\newunicodechar{↖}{\nwarrow} %
\newunicodechar{↗}{\nearrow} %
\newunicodechar{↝}{\ensuremath{\leadsto}}
\newunicodechar{↦}{\ensuremath{\mapsto}}
\newunicodechar{⇆}{\ensuremath{\leftrightarrows}} %
\newunicodechar{⇐}{\ensuremath{\Leftarrow}} %
\newunicodechar{⇒}{\ensuremath{\Rightarrow}} %
\newunicodechar{⇔}{\ensuremath{\Leftrightarrow}} %
\newunicodechar{∀}{\ensuremath{\forall}}   %
\newunicodechar{∂}{\ensuremath{\partial}}
\newunicodechar{∃}{\ensuremath{\exists}} %
\newunicodechar{∅}{\ensuremath{\varnothing}} %
\newunicodechar{∈}{\ensuremath{\in}}
\newunicodechar{∉}{\ensuremath{\not\in}} %
\newunicodechar{∋}{\ensuremath{\ni}}  %
\newunicodechar{∎}{\ensuremath{\qed}}%#+beamer_header: \newunicodechar{∎}{\ensuremath{\blacksquare}} % end of proof
\newunicodechar{∏}{\prod}
\newunicodechar{∑}{\sum}
\newunicodechar{∗}{\ensuremath{\ast}} %
\newunicodechar{∘}{\ensuremath{\circ}} %
\newunicodechar{∙}{\ensuremath{\bullet}} % #+beamer_header: \newunicodechar{∙}{\ensuremath{\cdot}}
\newunicodechar{⊙}{\ensuremath{\odot}}
\newunicodechar{∞}{\ensuremath{\infty}} %
\newunicodechar{∣}{\ensuremath{\mid}} %
\newunicodechar{∧}{\wedge}%
\newunicodechar{∨}{\vee}%
\newunicodechar{∩}{\ensuremath{\cap}} %
\newunicodechar{∪}{\ensuremath{\cup}} %
\newunicodechar{∫}{\int}
\newunicodechar{∷}{::} %
\newunicodechar{∼}{\ensuremath{\sim}} %
\newunicodechar{≃}{\ensuremath{\simeq}} %
\newunicodechar{≔}{\ensuremath{\coloneqq}} %
\newunicodechar{≅}{\ensuremath{\cong}} %
\newunicodechar{≈}{\ensuremath{\approx}} %
\newunicodechar{≜}{\ensuremath{\stackrel{\scriptscriptstyle {\triangle}}{=}}} % #+beamer_header: \newunicodechar{≜}{\triangleq}
\newunicodechar{≝}{\ensuremath{\stackrel{\scriptscriptstyle {\text{def}}}{=}}}
\newunicodechar{≟}{\ensuremath{\stackrel {_\text{\textbf{?}}}{\text{\textbf{=}}\negthickspace\negthickspace\text{\textbf{=}}}}} % or {\ensuremath{\stackrel{\scriptscriptstyle ?}{=}}}
\newunicodechar{≠}{\ensuremath{\neq}}%
\newunicodechar{≡}{\ensuremath{\equiv}}%
\newunicodechar{≤}{\ensuremath{\le}} %
\newunicodechar{≥}{\ensuremath{\ge}} %
\newunicodechar{⊂}{\ensuremath{\subset}} %
\newunicodechar{⊃}{\ensuremath{\supset}} %
\newunicodechar{⊆}{\ensuremath{\subseteq}} %
\newunicodechar{⊇}{\ensuremath{\supseteq}} %
\newunicodechar{⊎}{\ensuremath{\uplus}} %
\newunicodechar{⊑}{\ensuremath{\sqsubseteq}} %
\newunicodechar{⊒}{\ensuremath{\sqsupseteq}} %
\newunicodechar{⊓}{\ensuremath{\sqcap}} %
\newunicodechar{⊔}{\ensuremath{\sqcup}} %
\newunicodechar{⊕}{\ensuremath{\oplus}} %
\newunicodechar{⊗}{\ensuremath{\otimes}} %
\newunicodechar{⊛}{\ensuremath{\circledast}}
\newunicodechar{⊢}{\ensuremath{\vdash}} %
\newunicodechar{⊤}{\ensuremath{\top}}
\newunicodechar{⊥}{\ensuremath{\bot}} % bottom
\newunicodechar{⊧}{\models} %
\newunicodechar{⊨}{\models} %
\newunicodechar{⊩}{\Vdash}
\newunicodechar{⊸}{\ensuremath{\multimap}} %
\newunicodechar{⋁}{\ensuremath{\bigvee}}
\newunicodechar{⋃}{\ensuremath{\bigcup}} %
\newunicodechar{⋄}{\ensuremath{\diamond}} %
\newunicodechar{⋅}{\ensuremath{\cdot}}
\newunicodechar{⋆}{\ensuremath{\star}} %
\newunicodechar{⋮}{\ensuremath{\vdots}} %
\newunicodechar{⋯}{\ensuremath{\cdots}} %
\newunicodechar{─}{---}
\newunicodechar{■}{\ensuremath{\blacksquare}} % black square
\newunicodechar{□}{\ensuremath{\square}} %
\newunicodechar{▴}{\ensuremath{\blacktriangledown}}
\newunicodechar{▵}{\ensuremath{\triangle}}
\newunicodechar{▹}{\ensuremath{\triangleright}} %
\newunicodechar{▾}{\ensuremath{\blacktriangle}}
\newunicodechar{▿}{\ensuremath{\triangledown}}
\newunicodechar{◃}{\triangleleft{}}
\newunicodechar{◅}{\ensuremath{\triangleleft{}}}
\newunicodechar{◇}{\ensuremath{\diamond}} %
\newunicodechar{◽}{\ensuremath{\square}}
\newunicodechar{★}{\ensuremath{\star}}   %
\newunicodechar{♭}{\ensuremath{\flat}} %
\newunicodechar{♯}{\ensuremath{\sharp}} %
\newunicodechar{✓}{\ensuremath{\checkmark}} %
\newunicodechar{⟂}{\ensuremath{^\bot}} % PERPENDICULAR
\newunicodechar{⟦}{\ensuremath{\llbracket}}
\newunicodechar{⟧}{\ensuremath{\rrbracket}}
\newunicodechar{⦇}{\ensuremath{\llparenthesis}}
\newunicodechar{⦈}{\ensuremath{\rrparenthesis}}
\newunicodechar{⟨}{\ensuremath{\langle}} %
\newunicodechar{⟩}{\ensuremath{\rangle}} %
\newunicodechar{⌜}{\ensuremath{\ulcorner}} %
\newunicodechar{⌝}{\ensuremath{\urcorner}} %
\newunicodechar{⟶}{{\longrightarrow}}
\newunicodechar{⟷} {\ensuremath{\leftrightarrow}}
\newunicodechar{⟹}{\ensuremath{\Longrightarrow}} %
\newunicodechar{ⱼ}{_j} %
\newunicodechar{𝒟}{\ensuremath{\mathcal{D}}} %
\newunicodechar{𝒢}{\ensuremath{\mathcal{G}}}
\newunicodechar{𝒦}{\ensuremath{\mathcal{K}}} %
\newunicodechar{𝒫}{\ensuremath{\mathcal{P}}}
\newunicodechar{𝔸}{\ensuremath{\mathds{A}}} %
\newunicodechar{𝔹}{\ensuremath{\mathds{B}}} %
\newunicodechar{𝟙}{\ensuremath{\mathds{1}}} %
\newunicodechar{𝔼}{\mathds{E}}
\newunicodechar{⊬}{\not \vdash}
\newunicodechar{∝}{\propto}
\newunicodechar{ᶿ}{^\theta}
\institute[University of Rochester]{FACTS.lab, University of Rochester}
\hypersetup{
 pdfauthor={Julian Grove},
 pdftitle={Haskell: variables, data types, patterns, and recursion},
 pdfkeywords={},
 pdfsubject={},
 pdfcreator={Emacs 28.2 (Org mode 9.6.6)}, 
 pdflang={English}}
\begin{document}

\maketitle

\section{Introduction}
\label{sec:org9fb5694}

\begin{frame}[label={sec:orga52ce46}]{Last time}
We looked at some basic data types, functions, fixity, and type inference.
\end{frame}

\begin{frame}[label={sec:org9576599}]{This time}
We'll look at some of the stuff Haskell was made for.
\pause
\begin{itemize}[<+->]
\item Anonymous functions (i.e., function literals)
\item Role-your-own data types (i.e., algebraic data types)
\item Recursion via pattern matching
\end{itemize}
\end{frame}

\section{Variables}
\label{sec:org57238fb}
\begin{frame}[label={sec:org56df7d7},fragile]{\texttt{let} bindings}
 A \texttt{let} binding can be used to define a local variable anywhere you want.
\end{frame}

\begin{frame}[label={sec:org537dab7},fragile]{\texttt{where} clauses}
 A \texttt{where} clause can be used to define a local variable inside of another
definition.
\end{frame}

\begin{frame}[label={sec:orgef068dd},fragile]{Anonymous functions}
 Functions are first-class in Haskell, so they are treated like other data. \\[0pt]
\pause \bigskip
This means that we can write function \emph{literals}. \\[0pt]
\pause \bigskip
We do this by binding the variad to make \texttt{Fruit} an instance of the typeclass \texttt{Show}.
\begin{itemize}
\item Two ways to do this\ldots{}
\end{itemize}
\end{frame}

\begin{frame}[label={sec:orgb0b6e26},fragile]{Sum types}
 \texttt{Fruit} is what is called a \emph{sum type}.
\pause
\begin{itemize}[<+->]
\item It enumerates all values it can have in different branches, delimiting them
with a \texttt{|}.
\item In each branch is what is called a \emph{data constructor}.
\item The name of a data constructor in Haskell must begin with a capital letter.
\end{itemize}
\end{frame}

\begin{frame}[label={sec:org083e2e8},fragile]{N-ary constructors}
 The \texttt{Fruit} sum type is an odd special case, in that the data constructors
don't carry an extra data besides their identity. \\[0pt]
\pause \bigskip
Something more common might have data constructors carry additional data,
e.g., one data constructor could carry a \texttt{Bool} and one could carry a \texttt{String}.
\pause
\begin{itemize}[<+->]
\item This can allow us to write functions that take \emph{either} a \texttt{Bool} \emph{or} a \texttt{String} as
its input, using \emph{pattern matching}.
\end{itemize}
\pause \bigskip
By the way, you might be wondering: if a data constructor can take an
argument, does that mean it's a function?
\pause
\begin{itemize}[<+->]
\item The answer is ``yes''!
\end{itemize}
\end{frame}

\begin{frame}[label={sec:org9cfe40e},fragile]{Pattern matching: order matters}
 Pattern branches get checked in top-to-bottom order.
\pause
\begin{itemize}[<+->]
\item For example\ldots{}
\item Flipping the branches makes the definition effectively stop at the first
branch, since \texttt{str} is a wildcard over all possible strings.
\end{itemize}
\end{frame}

\begin{frame}[label={sec:orgf1269f4},fragile]{Case expressions}
 You can also use a \emph{case expression} to do pattern matching. \\[0pt]
\pause \bigskip
Case expressions do more than just pattern match---they also evaluate the
expression between the \texttt{case} and the \texttt{of}\ldots{}
\end{frame}

\begin{frame}[label={sec:org60a6096},fragile]{As patterns}
 An \emph{as pattern} (written with an \texttt{@} sign) allows you to bind an identifier to
the an argument which has been deconstructed into a pattern\ldots{}
\end{frame}

\begin{frame}[label={sec:org6820b80},fragile]{Pattern guards}
 Pattern guards are useful when you want to further restrict the applicability
of a branch of a definition to patterns that satisfy some boolean condition.
\\[0pt]
\pause \bigskip
You use a \texttt{|} after the relevant pattern and then state the condition\ldots{}
\end{frame}

\section{Recursive definitions}
\label{sec:orgb7c6e4d}
\begin{frame}[label={sec:org5cfc2b1}]{Lists}
We talked a little about lists last time. \\[0pt]
\pause \bigskip
Lists are deeply baked into Haskell, so we can't look at the source code. \\[0pt]
\pause \bigskip
But we can roll our own\ldots{}
\end{frame}

\begin{frame}[label={sec:org3b3c92a},fragile]{Haskell lists}
 For convenience, Haskell lets you type, e.g., \texttt{['a', 's', 'd', 'f']} for a list
literal. \\[0pt]
\pause \bigskip
When you see this, you should have in mind the following:
\begin{center}
\texttt{('a' : ('s' : ('d' : ('f' : []))))}
\end{center}
\pause \bigskip
Everything is one of two cases; either:
\pause
\begin{itemize}[<+->]
\item any empty list
\item something cons-ed onto a list
\end{itemize}
\end{frame}

\begin{frame}[label={sec:org1e8f36f},fragile]{Appending stuff}
 Let's define our first recursive function: \texttt{append}.
\end{frame}

\begin{frame}[label={sec:org40a047e},fragile]{\texttt{[a]} to \texttt{List a}}
 How could we write a recursive function that maps values of type \texttt{List a} to
values of type \texttt{[a]}?
\end{frame}

\begin{frame}[label={sec:org4c7cd6b},fragile]{\texttt{map}}
 Haskell has a built-in function \texttt{map} for mapping functions of type \texttt{a -> b} to
functions from lists of \texttt{a}'s to lists of \texttt{b}'s.
\begin{center}
\texttt{map :: (a -> b) -> [a] -> [b]}
\end{center}
How does \texttt{map} work?\ldots{}
\pause
\begin{itemize}[<+->]
\item We need a branch in the definition that applies to the empty list.
\item We need a branch in the definition that applies to non-empty lists.
\end{itemize}
\end{frame}

\begin{frame}[label={sec:orgb5db6ae},fragile]{\texttt{filter}}
 Filter takes a predicate, i.e., a function of from \texttt{a}'s to \texttt{Bool}'s, along with
a list of \texttt{a}'s, in order to give back a list of the \texttt{a}'s that satisfy the
predicate.
\begin{center}
\texttt{filter :: (a -> Bool) -> [a] -> [a]}
\end{center}
How does \texttt{filter} work?\ldots{}
\pause
\begin{itemize}[<+->]
\item We need a branch in the definition that applies to the empty list.
\item We need a branch in the definition that applies to non-empty lists.
\end{itemize}
\end{frame}

\begin{frame}[label={sec:orgac1012c},fragile]{\texttt{foldr} and \texttt{foldl}}
 Haskell has functions \texttt{foldr} and \texttt{foldl} that each take a two-place operation, a
starting value, and some list, in order to iteratively apply the function to
the elements of the list, one-by-one.
\begin{center}
\texttt{foldr :: (a -> b -> b) -> b -> [a] -> b} \\[0pt]
\texttt{foldl :: (b -> a -> b) -> b -> [a] -> b}
\end{center}
\end{frame}

\begin{frame}[label={sec:org465d58b},fragile]{\texttt{foldr}}
 \texttt{foldr}, in a way, conceptualizes a list as right-branching. \\[0pt]
\begin{center}
\begin{forest}
[{+} [ 7 ] [{+} [ 8 ] [{+} [ 9 ] [{+} [ 10 ] [ 0 ] ] ] ] ]
\end{forest}
\end{center}
\end{frame}

\begin{frame}[label={sec:orgdace915},fragile]{\texttt{foldl}}
 \texttt{foldl} conceptualizes it as left-branching. \\[0pt]
\begin{center}
\begin{forest}
[{+} [{+} [{+} [{+} [ 0 ] [ 7 ] ] [ 8 ] ] [ 9 ] ] [ 10 ] ] ] ] ]
\end{forest}
\end{center}
\end{frame}
\end{document}